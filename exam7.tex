\section{热力学统计物理随堂测验}
2023.4.23
\subsection{A组}
\begin{questions}
  \question 简述孤立系, 闭系和开系的概念.
  \begin{solution}
    (p3, 重考题)
    \begin{description}
      \item[孤立系统] 与其他物体既没有物质交换也没有能量交换的系统称为孤立系.
      \item[开系] 与外界既有物质交换, 又有能量交换的系统称为开系.
      \item[闭系] 与外界没有物质交换, 但有能量交换的系统称为闭系.
    \end{description}
  \end{solution}
  \question 简述热动平衡的概念.
  \begin{solution}
    (p4) 在平衡状态之下, 系统的宏观性质虽然不随时间改变, 但组成系统的大量微观粒子仍处在不断的运动之中, 只是这些微观粒子运动的统计平均效果不变而已, 因此热力学的平衡状态是一种动的平衡, 常称为热动平衡.
  \end{solution}
  \question 简述热力学第零定律.
  \begin{solution}
    (p7, 重考题)经验表明, 如果物体A和物体B各自与处在同意状态的物体C达到热平衡, 若令A与B进行热接触, 它们也将处在热平衡. 这个经验事实被称为热平衡定律(热力学第零定律).
  \end{solution}
  \question 简述热力学极限的概念.
  \begin{solution}
    (p14, 重考题) 将均匀系统所有的热力学量区分为强度量和广延量仅在系统所含粒子数$N\to\infty$, 体积$V\to\infty$而粒子数密度$\frac{N}{V}$为有限的极限情形才严格成立. 这一极限情形称为热力学极限. 对于通常的宏观物质系统($N\approx 10^{23}$), 上述特性无疑是很好的近似.
  \end{solution}
  \question 求理想气体卡诺循环热机和制冷机的效率.
  \begin{solution}
    (p29) 卡诺循环热机的效率为
    \begin{equation}
      \eta = 1-\frac{T_2}{T_1},
    \end{equation}
    制冷机的效率为
    \begin{equation}
      \eta' = \frac{T_2}{T_1-T_2}.
    \end{equation}
  \end{solution}
  \question 简述如何引入热力学温标.
  \begin{solution}
    (p35, 重考题)
    \begin{equation}
      \frac{Q_2}{Q_1} = \frac{T_2^*}{T_1^*},
    \end{equation}
    两个温度的比值是通过在这两个温度之间工作的可逆热机与热源交换的热量的比值来定义的. 由于比值$\frac{Q_2}{Q_1}$与工作物质的特性无关, 所引进的温标显然不依赖于任何具体物质的特性, 而是一种绝对温标, 称为热力学温标(也称开尔文温标).
  \end{solution}
  \question 简述统计物理中$\mu$空间的概念.
  \begin{solution}
    (p165) 用$q_1,\dots, q_r;p_1, \dots, p_r$ 共$2r$个变量为直角坐标, 构成一个$2r$维空间, 称为$\mu$空间.
  \end{solution}
\end{questions}

\subsection{B组}
\begin{questions}
  \question 简述热力学平衡态的概念.
  \begin{solution}
    (p3, 重考题)经验指出, 一个孤立系统, 不论其初态如何复杂, 经过足够长的时间后, 将会到达这样的状态, 系统的各种宏观性质在长时间内不发生任何变化, 这样的状态称为热力学平衡态.
  \end{solution}
  \question 简述状态参量和状态函数的概念.
  \begin{solution}
    (p4, 重考题)
    \begin{description}
      \item[状态参量]  我们可以根据问题的性质和考虑的方便选择其中几个宏观量作为自变量, 这些自变量本身可以独立的改变, 我们所研究的系统的其他宏观量又可以表达为它们的函数. 这些自变量就足以确定系统的平衡状态, 我们称它们为状态参量.
      \item[状态函数] 其他的宏观变量既然可以表达为状态参量的函数, 便称为状态函数.
    \end{description}
  \end{solution}
  \question 简述物态方程的概念.
  \begin{solution}
    (p8, 重考题) 物态方程就是给出温度与状态参量之间的函数关系的方程.
  \end{solution}
  \question 简述如何通过实验引入内能, 并给出内能的微观意义.
  \begin{solution}
    (p19) 焦耳进行了大量的实验, 发现用不同的绝热过程使物体升高一定的温度, 所需的功在实验误差范围内是相等的. 这就是说, 系统经绝热过程(包括非静态的绝热过程)从初态变到终态, 在过程中外界对系统所作的功仅取决于系统的初态和终态而与过程无关. 这个事实表明, 可以用绝热过程中外界对系统所作的功$W_S$定义一个态函数$U$在终态$B$和初态$A$之差:
    \begin{equation}
      U_B - U_A = W_S,
    \end{equation}
    态函数$U$ 称为内能.

    (p20)从微观的角度来看, 内能是系统中分子无规运动的能量总和的统计平均值. 无规运动的能量包括分子的动能, 分子间相互作用的势能以及分子内部运动的能量.
  \end{solution}
  \question 简述克劳修斯等式和不等式.
  \begin{solution}
    (p36) $Q_1$是从热源$T_1$吸取的热量, $Q_2$是从热源$T_2$吸取的热量, 则:
    \begin{equation}
      \frac{Q_1}{T_1} + \frac{Q_2}{T_2} \leq 0.
    \end{equation}
  \end{solution}
  \question 简述态密度的概念, 并计算三维自由粒子的态密度.
  \begin{solution}
    (p173)态密度$D(\varepsilon)$表示单位能量间隔内的可能状态数. 三维自由粒子的态密度为
    \begin{equation}
      D(\varepsilon) = \frac{2\pi V}{h^3}(2m)^{3/2}\varepsilon^{1/2}.
    \end{equation}
  \end{solution}
\end{questions}