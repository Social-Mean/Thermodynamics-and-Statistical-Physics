\section{热力学统计物理随堂测验}
2023.4.3
\subsection{A组}
\begin{questions}
  \question 简述临界点的概念.
  \begin{solution}
    (p99) 连续相变的相变点也称为临界点.
  \end{solution}
  \question 简述一级相变的概念.
  \begin{solution}
    (p97) 一级相变指的是相变点两相的化学势连续, 但化学势的一级偏导数存在突变:
    \begin{equation}
      \begin{aligned}
         & \mu^{(1)}(T,p) = \mu^{(2)}(T,p).                                                    \\
         & \pdv{\mu^{(1)}}{T}\neq\pdv{\mu^{(2)}}{T}, \pdv{\mu^{(1)}}{p}\neq\pdv{\mu^{(2)}}{p}.
      \end{aligned}
    \end{equation}
  \end{solution}
  \question 简述孤立系处在稳定平衡状态的必要和充分条件.
  \begin{solution}
    (p76)
    \begin{equation}
      \Delta S < 0.
    \end{equation}
  \end{solution}
  \question 证明下列平衡判据: 在$F, V$不变的情况下, 稳定平衡态的$T$最小.
  \begin{solution}
    (p106, 3.1(d))由自由能的定义$F=U-TS$和热力学第二定律的数学表达式$\delta U < T\delta S + \dj W$, 在虚变动中必有:
    \begin{equation}
      \delta F < -S\delta T +\dj W.
    \end{equation}
    在$F$和$V$不变的情形下, 有
    \begin{equation}
      \delta F = 0, \dj W = 0,
    \end{equation}
    故在变动中必有
    \begin{equation}
      S\delta T <0.
    \end{equation}
    由于$S>0$, 如果系统达到了$T$为极小的状态, 它的温度不可能再降低, 系统就不可能自发发生任何宏观的变化性而处在稳定的平衡状态, 因此, 在$F$, $V$不变的情形下, 稳定平衡态的$T$最小.
  \end{solution}
\end{questions}
\subsection{B组}
\begin{questions}
  \question 简述临界现象的概念.
  \begin{solution}
    (p99) 临界现象指物质在连续相变临界点邻域的行为.
  \end{solution}
  \question 简述连续相变的概念.
  \begin{solution}
    (p99)非一级相变称为连续相变.
  \end{solution}
  \question 简述等温等容系统处在稳定平衡状态的必要和充分条件.
  \begin{solution}
    (p77)
    \begin{equation}
      \Delta F > 0.
    \end{equation}
  \end{solution}
  \question 证明下列平衡判据: 在$G, p$ 不变的情况下, 稳定平衡态的$T$最小.
  \begin{solution}
    (p106, 3.1(e)) 与A组第4题类似.
  \end{solution}
\end{questions}