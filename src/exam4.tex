\section{统计物理随堂测验}
2023.3.27
\subsection{A组}
\begin{questions}
  \question 简述等温等压系统处在稳定平衡状态的必要和充分条件.
  \begin{solution}
    (p77)吉布斯函数判据
    \begin{equation}
      \Delta G > 0 .
    \end{equation}
  \end{solution}
  \question 证明单元系和复相平衡条件.
  \begin{solution}
    (p83)单元系复相平衡条件为
    \begin{equation}
      \left\{\begin{aligned}
         & T^\alpha = T^\beta\qc     & \text{(热平衡条件)},  \\
         & p^\alpha = p^\beta\qc     & \text{(力学平衡条件)}, \\
         & \mu^\alpha = \mu^\beta\qc & \text{(相变平衡条件)}.
      \end{aligned}\right.
    \end{equation}
    证明如下:

    指标$\alpha$和$\beta$表示两个相, 整个系统是孤立系统, 它的总内能, 总体积和总物质的量应是恒定的, 即
    \begin{equation}
      \left\{\begin{aligned}
        U^\alpha + U^\beta = \text{常量}, \\
        V^\alpha + V^\beta = \text{常量}, \\
        n^\alpha + n^\beta = \text{常量}.
      \end{aligned}\right.
    \end{equation}
    设想系统发生一个虚变动. 在虚变动中, 各虚变动满足
    \begin{equation}
      \left\{\begin{aligned}
        \delta U^\alpha +\delta U^\beta =0, \\
        \delta V^\alpha +\delta V^\beta =0, \\
        \delta n^\alpha +\delta n^\beta =0.
      \end{aligned}\right.
    \end{equation}
    由于
    \begin{equation}
      \dd U = T\dd S - p\dd V + \mu\dd n,
    \end{equation}
    所以
    \begin{equation}
      \left\{\begin{aligned}
         & \delta S^\alpha = \frac{\delta U^\alpha+p^\alpha\delta V^\alpha-\mu^\alpha\delta n^\alpha}{T^\alpha}, \\
         & \delta S^\beta = \frac{\delta U^\beta+p^\beta\delta V^\beta-\mu^\beta\delta n^\beta}{T^\beta}.
      \end{aligned}\right.
    \end{equation}
    根据熵的广延性质, 整个系统的熵变是
    \begin{equation}
      \begin{aligned}
        \delta S & = \delta S^\alpha + \delta S^\beta                          \\
                 & = \delta U^\alpha\qty(\frac{1}{T^\alpha}-\frac{1}{T^\beta})
        + \delta V^\alpha
        \qty(\frac{p^\alpha}{T^\alpha}-\frac{p^\beta}{T^\beta})
        - \delta n^\alpha\qty(\frac{\mu^\alpha}{T^\alpha}-\frac{\mu^\beta}{T^\beta}).
      \end{aligned}
    \end{equation}
    整个系统达到平衡时, 总熵有极大值, 必有
    \begin{equation}
      \delta S = 0.
    \end{equation}
    因为 $\delta U^\alpha, \delta V^\alpha, \delta n^\alpha$ 是可以独立改变的, $\delta S=0$, 要求
    \begin{equation}
      \left\{\begin{aligned}
         & \frac{1}{T^\alpha}-\frac{1}{T^\beta} = 0,                  \\
         & \frac{p^\alpha}{T^\alpha}-\frac{p^\beta}{T^\beta} = 0,     \\
         & \frac{\mu^\alpha}{T^\alpha}-\frac{\mu^\beta}{T^\beta} = 0.
      \end{aligned}\right.
    \end{equation}
    即
    \begin{equation}
      \left\{\begin{aligned}
         & T^\alpha = T^\beta\qc     & \text{(热平衡条件)},  \\
         & p^\alpha = p^\beta\qc     & \text{(力学平衡条件)}, \\
         & \mu^\alpha = \mu^\beta\qc & \text{(相变平衡条件)}.
      \end{aligned}\right.
    \end{equation}
  \end{solution}
  \question 证明在等温过程中, 系统对外界所做的功 $-W$ 不大于其自由能的减少.
  \begin{solution}
    (p46) 等温过程中,
    \begin{equation}
      Q \leq T(S_B - S_A),
    \end{equation}
    根据热力学第一定律, $U_B-U_A=W+Q$. 代入得
    \begin{equation}
      -W \leq (U_A - U_B) - T(S_A - S_B).
    \end{equation}
    自由能为
    \begin{equation}
      F = U-TS,
    \end{equation}
    所以
    \begin{equation}
      -W \leq F_A - F_B,
    \end{equation}
    即系统在等温过程中对外所作的功不大于其自由能的减小.
  \end{solution}
  \question 对于均匀简单系统, 由已知的吉布斯函数$G$, 给出熵, 物态方程, 内能和焓.
  \begin{solution}
    (p63)
\begin{gather}
  S = -\pdv{G}{T},\\
  V = \pdv{G}{p}, \\
  U = G - T\pdv{G}{T} - p\pdv{G}{p},\\
  H = G - T\pdv{G}{T}.
\end{gather}
(注意$G=U-TS+pV$, 不要跟自由能$F$搞混了.)
  \end{solution}
  \question 对于均匀简单系统, 证明:
  \begin{equation}
    C_p = T\qty(\pdv{S}{T})_p,
  \end{equation}
  和
  \begin{equation}
    \qty(\pdv{U}{V})_T = T\qty(\pdv{S}{V})_T - p.
  \end{equation}
  \begin{solution}
(p54, p53)
\begin{equation}
  C_p = \qty(\pdv{H}{T})_p
\end{equation}
写出$\dd H$的表达式, 把$S$看成$T,p$的函数, 并求全微分.

写出$\dd U$的表达式, 把$S$看成$V,T$的函数, 并求全微分.
  \end{solution}
\end{questions}

\subsection{B组}
\begin{questions}
  \question 简述等温等容系统处在稳定平衡状态的必要和充分条件.
  \begin{solution}
    \begin{equation}
      \Delta F > 0.
    \end{equation}
  \end{solution}
  \question 写出开系的热力学基本方程, 并给出开系的特性函数的全微分形式.
  \begin{solution}
    开系的热力学基本方程为
    \begin{equation}
      \dd U = T\dd S - p\dd V + \mu\dd n.
    \end{equation}
    开系的特性函数的全微分形式为
    \begin{equation}
      \dd G = -S\dd T + V\dd p +\mu\dd n.
    \end{equation}
    \begin{equation}
      \dd H = T\dd S + V\dd p + \mu\dd n.
    \end{equation}
    \begin{equation}
      \dd F = -S\dd T - p\dd V + \mu\dd n.
    \end{equation}
    巨热力势$J=F-\mu n$, 全微分为
    \begin{equation}
      \dd J = -S\dd T - p\dd V - n\dd \mu.
    \end{equation}
  \end{solution}
  \question 证明, 若只有体积功时, 等温等压过程后, 吉布斯函数永不增加.
  \begin{solution}
(p47)
\begin{gather}
  -W \le F_A - F_B,\\
  W = -p(V_B-V_A),\\
  G = F+pV,\\
  G_B-G_A\le 0.
\end{gather}
  \end{solution}

  \question 对于均匀简单系统, 由已知的自由能$F$, 给出熵, 物态方程和内能.
  \begin{solution}
(p63) 与A组题类似, 只是这里换成了自由能. $F = U-TS$.
  \end{solution}
  \question 对于均匀简单系统, 证明:
  \begin{equation}
    C_V = T\qty(\pdv{S}{T})_V,
  \end{equation}
  和
  \begin{equation}
    \qty(\pdv{H}{p})_T = T\qty(\pdv{S}{p})_T + V.
  \end{equation}

  \begin{solution}
(p53) 与A组题类似,
\begin{equation}
  C_V = \qty(\pdv{U}{T})_V.
\end{equation}
写出$\dd U$的表达式, 把$S$看成$T,V$的函数, 并求全微分.

写出$\dd H$的表达式, 把$S$看成$T,p$的函数, 并求全微分.
  \end{solution}
\end{questions}