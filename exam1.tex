\section{统计物理随堂测验}
2023.3.6
\subsection{A组}
\begin{questions}
  \question 简述热力学的研究对象和统计物理的优缺点.
  \begin{solution}
    热力学研究的对象是由大量微观粒子(分子或其他粒子)组成的宏观物质系统.

    统计物理的优缺点(\textcolor{red}{待补充}).
  \end{solution}
  \question 简述孤立系统, 开系和闭系的概念.
  \begin{solution}
    \begin{description}
      \item[孤立系统] 与其他物体既没有物质交换也没有能量交换的系统称为孤立系.
      \item[开系] 与外界既有物质交换, 又有能量交换的系统称为开系.
      \item[闭系] 与外界没有物质交换, 但有能量交换的系统称为闭系.
    \end{description}
  \end{solution}
  \question 简述状态参量和状态函数的概念.
  \begin{solution}
    \begin{description}
      \item[状态参量]  我们可以根据问题的性质和考虑的方便选择其中几个宏观量作为自变量, 这些自变量本身可以独立的改变, 我们所研究的系统的其他宏观量又可以表达为它们的函数. 这些自变量就足以确定系统的平衡状态, 我们称它们为状态参量.
      \item[状态函数] 其他的宏观变量既然可以表达为状态参量的函数, 便称为状态函数.
    \end{description}
  \end{solution}
  \question 简述热平衡定律
  \question 简述准静态过程的概念.
  \question 简述热力学极限的概念.
\end{questions}
\subsection{B组}
\begin{questions}
  \question 简述统计物理的研究对象和热力学的优缺点.
  \question 简述平衡态的概念.
  \question 简述简单系统, 均匀系和相的概念.
  \question 简述物态方程的概念.
  \question 如何引入热量的概念.
  \question 写出准静态过程中外界对系统做功的通式, 结合该通式简述做功和传热对系统改变的不同.
\end{questions}